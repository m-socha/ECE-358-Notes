\documentclass[12pt,titlepage]{article}
\usepackage[margin=1in]{geometry}

\begin{document}
  \begin{titlepage}
    \vspace*{\fill}
    \centering

    \textbf{\Huge ECE 358 Course Notes} \\ [0.4em]
    \textbf{\Large Computer Networks} \\ [1em]
    \textbf{\Large Michael Socha} \\ [1em]
    \textbf{\large 4A Software Engineering} \\
    \textbf{\large University of Waterloo} \\
    \textbf{\large Spring 2018} \\
    \vspace*{\fill}
  \end{titlepage}

  \newpage 

  \tableofcontents

  \newpage

  \section{Course Overview}
    \subsection{Logistics}
      \begin{itemize}
        \item \textbf{Professor:} Albert Wasef
        \item \textbf{Email:} awasef@uwaterloo.ca
        \item \textbf{Phone:} ext. 31723
        \item \textbf{Office:} EIT-4012
      \end{itemize}

    \subsection{Topics Covered}
      This course focuses on the fundamentals of networking and thinking like a network engineer.
      Specific topics covered by this course include:
      \begin{itemize}
        \item LAN technologies and underlying protocols
        \item Transport protocols (TCP, retransmission)
        \item IP layer concepts (e.g. routing, addressing)
        \item Discrete-event simulation
        \item Network utilities
      \end{itemize}

  \section{Introduction - Internet Fundamentals}
    \subsection{What is the Internet?}
      The Internet is the world's largest computer network, connecting billions of devices. Devices
      connected to the Internet are known as hosts (end systems), and are running some kinds of network applications.
      Communication links are necessary for hosts to share information with each other, which can be done through
      a variety of means, including cables (e.g. fiber, copper), radio, or satellite. Packet switches (e.g. routers,
      switches) are responsible for forwarding chunks of data through these communication links.

      Standardized protocols are necessary for communication between hosts. Protocols define the format and order of
      messages sent as well as actions taken upon message transmission and reception. Sample protocols include TCP, IP, and HTTP.
      These standards are maintained by the IEFT (Internet Engineering Task Force).

      The Internet can also be viewed from a more service-oriented perspective, since it can be used to provide
      services such as the Web, VoIP, email, etc. to applications. The Internet also provides a programming interface
      to applications to interact with connected hosts.

\end{document}

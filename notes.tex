\documentclass[12pt,titlepage]{article}
\usepackage[margin=1in]{geometry}

\begin{document}
  \begin{titlepage}
    \vspace*{\fill}
    \centering

    \textbf{\Huge ECE 358 Course Notes} \\ [0.4em]
    \textbf{\Large Computer Networks} \\ [1em]
    \textbf{\Large Michael Socha} \\ [1em]
    \textbf{\large 4A Software Engineering} \\
    \textbf{\large University of Waterloo} \\
    \textbf{\large Spring 2018} \\
    \vspace*{\fill}
  \end{titlepage}

  \newpage 

  \tableofcontents

  \newpage

  \section{Course Overview}
    \subsection{Logistics}
      \begin{itemize}
        \item \textbf{Professor:} Albert Wasef
        \item \textbf{Email:} awasef@uwaterloo.ca
        \item \textbf{Phone:} ext. 31723
        \item \textbf{Office:} EIT-4012
      \end{itemize}

    \subsection{Topics Covered}
      This course focuses on the fundamentals of networking and thinking like a network engineer.
      Specific topics covered by this course include:
      \begin{itemize}
        \item LAN technologies and underlying protocols
        \item Transport protocols (TCP, retransmission)
        \item IP layer concepts (e.g. routing, addressing)
        \item Discrete-event simulation
        \item Network utilities
      \end{itemize}

  \section{Introduction - Internet Fundamentals}
    \subsection{What is the Internet?}
      The Internet is the world's largest computer network, connecting billions of devices. Devices
      connected to the Internet are known as hosts (end systems), and are running some kinds of network applications.
      Communication links are necessary for hosts to share information with each other, which can be done through
      a variety of means, including cables (e.g. fiber, copper), radio, or satellite. Packet switches (e.g. routers,
      switches) are responsible for forwarding chunks of data through these communication links.

      Standardized protocols are necessary for communication between hosts. Protocols define the format and order of
      messages sent as well as actions taken upon message transmission and reception. Sample protocols include TCP, IP, and HTTP.
      These standards are maintained by the IEFT (Internet Engineering Task Force).

      The Internet can also be viewed from a more service-oriented perspective, since it can be used to provide
      services such as the Web, VoIP, email, etc. to applications. The Internet also provides a programming interface
      to applications to interact with connected hosts.

    \subsection{Network Edges}
      Edge devices provide some sort of entry point to a network. Examples include computers, mobile devices, and servers
      (often in data centers). Communication between devices on a network can be wired or wireless.

      End systems can connect to an edge router through various ways, including using residential access nets, institutional
      access networks and mobile access networks. Important considerations in such connections include a connection's bandwith,
      latency, and whether it is shared/dedicated.

      Network connections can be made through a digital subscriber line (DSL), which allows for the transmission of data over
      telephone lines. A digital subscriber line access multiplexer (DSLAM) can be used to connect multiple DSL lines to a digital
      communications channel. Downstream transmission rates (typically < 1 Mbps) tend to be much faster than upstream transmission
      rates (typically < 10 Mbps). Optimal transmission rates are rarely reached in practice. Each line connects directly to a
      central office.

      Network connections can also be made through a cable network, which uses the same infrastructure as cable television.
      Differing frequencies are used to distinguish between different channels of communication. Hybrid Coaxial Cables (HFCs) are
      used to form the connection, which tend to have a downstream transmission around 30 Mbps and an upstream transmission around
      2 Mbps. These connections attach to an ISP router, and multiple parties typically share access to a cable headend.

      Most enterprise access networks use Ethernet connections, which tend to be much faster (available speeds include 10 Mbps, 100 Mbps,
      1 Gbps, 10 Gbps). Nowadays, most end systems connect to an Ethernet switch.

      Wireless access networks can connect end systems to routers without a cable connection. Wireless LANs (i.e. Wi-Fi) provide network
      access for a fairly small range, while wide-area access networks are provided by cellular operators and have a range of 10s of
      kilometers. Wireless LANs tend to have higher bandwidths than wide-area access networks.

      A host sending function is responsible for:
        \begin{itemize}
          \item Taking an application message
          \item Breaking the message into chunks (packets) or length $L$ bits
          \item Transmiting packets across a network at transmission rate $R$
        \end{itemize}
    $packet transmission delay = \frac{L}{R}$

    The medium facilitating transmission between a transmitter and receiver is called a physical link. Physical links may be guided
    (i.e. solid cables, such as copper, fiber or coax), or unguided (i.e. signals may propogate freely, such as through radio).

    Coaxial cables are formed from two concentric copper conductors. Coaxial cables bidirectional and they are are broadband, so they
    can support communication across multiple channels.

    Fiber optic cables feature a glass fiber carrying light pulses, where each pulse represents one bit. Fiber optics cables support
    high-speed point-to-point transmission, and have a low error rate.

    Radio is a wireless bidirectional signal carried in the electromagnetic spectrum. The environment of propogation may cause signal
    reflection, obstruction (by objects in path) and interference. Radio link network types include terrestrial microwaves, LAN, wide-area
    and satellite.

    \subsection{Network Core}
    Through store-and-forward packet-switching, an entire packet must arrive at a router before it can be transmitted on the next link. The
    resulting end-end delay is $\frac{2L}{R}$ (plus any propogation delay).

    Should the arrival rate exceed a link's transmission rate, the resulting packets will queue up. If the memory in which the packets are
    stored fills up, packets can be dropped.

    Routing determines the source-destination route taken by packets, while forwarding moves packets to the appropriate output router.

    Circuit switching is an alternative design for a network core. Instead of queueing up packets along shared lines, end-end resources
    between a transmission's source and destination are reserved (i.e. circuitry used only for that specific transmission). Such an approach
    is commonly used in telephone networks. Circuit switching can be implemented using FDM (frequency-division multiplexing) or TDM
    (time-division multiplexing).

\end{document}
